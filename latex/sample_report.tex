% Latex template for MA305 Project Report, Spring 2015
%%%%%%%%%%%%%%%%%%%%%%%%%%%%%%%%%%%%%%%%%%%%%%%%%%%%%%%%%%%
\documentclass[11pt]{article}
\usepackage{graphicx}
\usepackage{color}
\newcommand{\cred} {\textcolor{red}}
\usepackage{fancyhdr}
\newcommand{\horrule}[1]{\rule{\linewidth}{#1}} 	% Horizontal rule

\begin{document}

%%%%%%%%%% TITLE PAGE %%%%%%%%%%
\begin{center}
{\it MA305, Spring 2017 \hfill Embry-Riddle Aeronautical University, FL 
 }
		\horrule{0.5pt} \\[0.4cm]
		{\bf \Large  Final Project Report}\\
		\horrule{2pt} \\[2cm]
%%%%%%%%%%%%%%%%%%%%
Your Name % change this
\\[0.4cm]
\today % change this
\end{center}
\thispagestyle{empty}
\newpage
\begin{abstract}

\end{abstract}
\tableofcontents 
\newpage

%%%%%%%%%%  %%%%%%%%%%
\section{Introduction}\label{S:1}
%The text of this section.
Write a brief description (background/significance) of what the project is about. 


\section{Problem Statement and Assumptions}\label{S:2}
%The text of this section. 
State fully and precisely the mathematical problem. 
State any assumptions made for the formulation of the model. 
Explain meaning of all symbols used. Make clear what is given and what we are looking for. 

\subsection{Part I}\label{S:2.1}
%
Text introducing this subsection. 

\subsection{Part II}\label{S:2.2}
%
Text introducing this subsection.

\section{Method/Analysis}\label{S:3}
%Text introducing this section
Begin with naming or characterizing the method/approach to be used, perhaps explain the basic idea behind it, to what type of problems it applies, under what conditions, what it achieves, what are its main features, advantages, disadvantages. Justify why it is applicable to this problem, stating clearly any assumptions you need to make about the problem for the method to apply. Name some other methods/approaches one could use, and if/why your method may be preferable.


\section{Solutions/Results}\label{S:4}
%Text introducing this section
This section contains the presentation of your solution and results.
Describe your implementation of the method(s) for this specific problem, any special features, numerical methods implementation  strategy, choices of any parameters, stopping criteria, etc.
Present the results in words and plots (annotate by hand if necessary), explain what they mean. Include your code in an Appendix. 

\subsection{A subsection}
%
Text introducing this subsection. 

\subsubsection{A subsubsection}
%
Text introducing this subsubsection. 

\subsubsection{A further subdivision}
%
Text introducing this subsubsection. 

\section{Discussion/Conclusions}\label{S:5}
%Text introducing this subsection
Interpret your solution physically, what we learn from it, comment on strengths and weaknesses of the solution method, any nice features you want to brag about, possible ways to improve it (e.g. how to make it more accurate, more efficient), as appropriate.


\begin{thebibliography}{100}
\bibitem{a1}  
Heath, Michael T., Scientific Computing: An Introductory Survey, McGraw Hill, 2002.
%
%\bibitem{a2}
%
%\bibitem{a3} 

\end{thebibliography}
\end{document}



%%%%%%%%%%%%%%%%%%%%%%%%%%%%%% section Appendix %%%%%%%%%%%%%%%%%%%%%
\appendix 
\setcounter{section}{0}           
\section{Codes and Makefiles}\label{S:A}
%
Text introducing the/this appendix.

Subsections and further divisions can also be used in appendices.

%%%%%%%%%% BIBLIOGRAPHY %%%%%%%%%%

\end{document}

\grid
