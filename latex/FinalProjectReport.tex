% Latex template for MA305 Project Report, Spring 2015
%%%%%%%%%%%%%%%%%%%%%%%%%%%%%%%%%%%%%%%%%%%%%%%%%%%%%%%%%%%
\documentclass[11pt]{article}
\usepackage{graphicx}
\usepackage{color}
\newcommand{\cred} {\textcolor{red}}
\usepackage{fancyhdr}
\newcommand{\horrule}[1]{\rule{\linewidth}{#1}} 	% Horizontal rule

\begin{document}

%%%%%%%%%% TITLE PAGE %%%%%%%%%%
\begin{center}
{\it MA305, Spring 2017 \hfill Embry-Riddle Aeronautical University, FL 
 }
		\horrule{0.5pt} \\[0.4cm]
		{\bf \Large  Newton Iteration and Fractals}\\
		\horrule{2pt} \\[2cm]
%%%%%%%%%%%%%%%%%%%%
Kyle Latino \& Thomas Levine
\\[0.4cm]
23 April 2017 % change this
\end{center}
\thispagestyle{empty}
\newpage
\begin{abstract}

\end{abstract}
\tableofcontents 
\newpage

%%%%%%%%%%  %%%%%%%%%%
\section{Introduction}\label{S:1}
%The text of this section.
Fractals are interesting and arguably beautiful representations of repetition at varying scales. Some notable fractals are the Serpensky Triangle or the Menger Sponge. These figures have some interesting characteristics such as visibly repetitive patterns at any zoom and theoretically infinite surface area. The Menger Sponge can be found in modern wireless devices as its massive surface area allows it to be used as an antenna while maintaining a very compact size.


\section{Problem Statement and Assumptions}\label{S:2}
%The text of this section. 
State fully and precisely the mathematical problem. 
State any assumptions made for the formulation of the model. 
Explain meaning of all symbols used. Make clear what is given and what we are looking for. 


\subsection{Newton Method for Root Finding}\label{S:2.1}
%Text introducing this subsection. 
In 1669 Sir Iaasic Newton developed a method for determining the roots of an equation. The pretense is: given an equation and an initial estimate for the root, you can iteratively approach the true root of the function. This method is superior to the simple bisection method in that it not only requires fewer input variables but also reaches the true value in fewer iterations. 

\subsection{Fractals from Newton Iterations}\label{S:2.2}
%Text introducing this subsection.
Using the Newton Method for root finding we will generate points and plot them using the open source tool gnuplot.

\section{Method/Analysis}\label{S:3}
%Text introducing this section
Begin with naming or characterizing the method/approach to be used, perhaps explain the basic idea behind it, to what type of problems it applies, under what conditions, what it achieves, what are its main features, advantages, disadvantages. Justify why it is applicable to this problem, stating clearly any assumptions you need to make about the problem for the method to apply. Name some other methods/approaches one could use, and if/why your method may be preferable.


\section{Solutions/Results}\label{S:4}
%Text introducing this section
Using a Ubuntu terminal we remotely connected to the WXSession server via a secure shell and 

\subsection{Newton Method}
%
The guess for the root of the function converges onto the true root within 6 iterations when using the Newton Method. Conversely, the bisection method requires 37 iterations to converge. This shows that for the function \[F(x)=x^3+x^2-3x-3\] the Newton method is able to find the root in 1/6th the time required by the bisection method.

Function roots: -1.7320508075688772, -1, 1.7320508075688774
\subsubsection{Newton's Cubit}
%
Newton famously used this method to solve only one function known as Newton's Cubit. \[F(x)=x^3-2x-5\]
Function root: 2.0945514815423265  

\subsubsection{A further subdivision}
%
Text introducing this subsubsection. 

\section{Discussion/Conclusions}\label{S:5}
%Text introducing this subsection
Interpret your solution physically, what we learn from it, comment on strengths and weaknesses of the solution method, any nice features you want to brag about, possible ways to improve it (e.g. how to make it more accurate, more efficient), as appropriate.


\begin{thebibliography}{100}
\bibitem{a1}  
Heath, Michael T., Scientific Computing: An Introductory Survey, McGraw Hill, 2002.
%
%\bibitem{a2}
%
%\bibitem{a3} 

\end{thebibliography}
\end{document}



%%%%%%%%%%%%%%%%%%%%%%%%%%%%%% section Appendix %%%%%%%%%%%%%%%%%%%%%
\appendix 
\setcounter{section}{0}           
\section{Codes and Makefiles}\label{S:A}
%
Text introducing the/this appendix.

Subsections and further divisions can also be used in appendices.

%%%%%%%%%% BIBLIOGRAPHY %%%%%%%%%%

\end{document}

\grid
